\documentclass[a4paper]{article}
\usepackage{amsmath}
\usepackage{bm}
\usepackage{titlesec}
\titleformat{\section}[block]{\LARGE\bfseries}{Problem 1.\Alph{section}}{1em}{}[]
\titleformat{\subsection}[block]{\normalsize\bfseries}{\arabic{subsection}.}{1em}{}[]
\titleformat{\paragraph}[block]{\small\bfseries}{[\arabic{paragraph}]}{1em}{Solution:}[]

\title{Chapter 1}
\author{HammerLi}
\date{\today}

\begin{document}
\maketitle

\section{}
\subsection{}
Suppose $a$ and $b$ are real numbers, not both $0$ . Find real numbers $c$ and $d$ such that
$$
    \frac{1}{a+bi} = c+di.
$$
\paragraph{}
$$
    \frac{1}{a+bi} = \frac{a-bi}{(a+bi)(a-bi)} = \frac{a-bi}{a^2+b^2} = c+di
    \Rightarrow \begin{cases}
        c = \frac{a}{a^2+b^2} \\
        d = -\frac{b}{a^2+b^2}
    \end{cases}
$$

\subsection{}
Show that
$$
    \frac{-1+\sqrt{3}i}{2}
$$
is a cube root of 1(meaning that its cube equals 1).
\paragraph{}
Simply, we have
$$
    \begin{aligned}
        (\frac{-1+\sqrt{3}i}{2})^3
         & = \frac{(-1+\sqrt{3}i)^3}{8}              \\
         & = \frac{(-2-2\sqrt{3}i)(-1+\sqrt{3}i)}{8} \\
         & = \frac{2(1+\sqrt{3}i)(1-\sqrt{3}i)}{8}   \\
         & = \frac{2 \times (1+3)}{8}                \\
         & = 1
    \end{aligned}
$$

More tricky solution is as below which depends on the computation methods of n-th root of a complex.
$$
    \begin{aligned}
        s & = \sqrt[3]{1}                                   \\
          & = \sqrt[3]{1+0 \cdot i}                         \\
          & = \sqrt[3]{\cos{0}+i\sin{0}}                    \\
          & = \cos{\phi}+i\sin{\phi}                        \\
          & (\phi = \frac{1}{3}(0+2k\pi) \quad k = 0, 1, 2)
    \end{aligned}
    \Rightarrow s = 1, \frac{-1+\sqrt{3}i}{2}, \frac{-1-\sqrt{3}i}{2}
$$

\subsection{}
Find two distinct square roots of $i$.
\paragraph{}
Use the same tricky method shown above.
$$
    \begin{aligned}
        s & = \sqrt{i}                                               \\
          & = \sqrt{0+1 \cdot i}                                     \\
          & = \sqrt{\cos{\frac{\pi}{2}} + i\sin{\frac{\pi}{2}}}      \\
          & = \cos{\phi} + i\sin{\phi}                               \\
          & (\phi = \frac{1}{2}(\frac{\pi}{2}+2k\pi) \quad k = 0, 1)
    \end{aligned}
    \Rightarrow s = \frac{\sqrt{2}(1+i)}{2},\frac{-\sqrt{2}(1+i)}{2}
$$

For another solution, let $i = (a+bi)^2$, we have
$$
    \begin{aligned}
        i & = (a+bi)^2         \\
          & = a^2 - b^2 + 2abi \\
    \end{aligned}
    \Rightarrow
    \begin{cases}
        a^2 - b^2 = 0 \\
        2ab = 1       \\
    \end{cases}
    \Rightarrow
    a = b = \pm\frac{\sqrt{2}}{2}
$$
And we will get the same answer.

\subsection{}
Show that $\alpha + \beta = \beta + \alpha$ for all $\alpha, \beta \in \bm{C}.$
\paragraph{}
Let $\alpha = a + bi, \beta = c + di$ for all $a,b,c,d \in \bm{R}$, we have
$$
    \begin{aligned}
        \alpha + \beta    & = (a + bi) + (c + di)           \\
                          & = (a + c) + (b + d)i            \\
        \beta + \alpha    & = (c + di) + (a + bi)           \\
                          & = (c + a) + (d + b)i            \\
                          & = (a + c) + (b + d)i            \\
        \Rightarrow \quad & \alpha + \beta = \beta + \alpha
    \end{aligned}
$$

\subsection{}
Show that $(\alpha + \beta) + \lambda = \alpha + (\beta + \lambda)$ for all $\alpha, \beta, \lambda \in \bm{C}$.
\paragraph{}
Let $\alpha = a + bi, \beta = c + di, \lambda = e + fi$ for all $a,b,c,d,e,f \in \bm{R}$, we have
$$
    \begin{aligned}
        (\alpha + \beta) + \lambda & = ((a + bi) + (c + di)) + (e + fi)                      \\
                                   & = ((a + c) + (b + d)i) + (e + fi)                       \\
                                   & = (a + c + e) + (b + d + f)i                            \\
        \alpha + (\beta + \lambda) & = (a + bi) + ((c + di) +(e + fi))                       \\
                                   & = (a + bi) + ((c + e) + (d + f)i)                       \\
                                   & = (a + c + e) + (b + d + f)i                            \\
        \Rightarrow        \quad   & (\alpha + \beta) + \lambda = \alpha + (\beta + \lambda)
    \end{aligned}
$$

\subsection{}
Show that $(\alpha\beta)\lambda=\alpha(\beta\lambda)$for all $\alpha,\beta,\lambda\in\bm{C}$.
\paragraph{}
Let $\alpha = a + bi, \beta = c + di, \lambda = e + fi$ for all $a,b,c,d,e,f \in \bm{R}$, we have
$$
    \begin{aligned}
        (\alpha\beta)\lambda     & = ((a + bi)(c + di))(e + fi)                         \\
                                 & = ((ac - bd) + (ad + bc)i)(e + fi)                   \\
                                 & = (ace - bde - adf - bcf) + (ade + bce + acf - bdf)i \\
        \alpha(\beta\lambda)     & = (a + bi)((c + di)(e + fi))                         \\
                                 & = (a + bi)((ce - fd) + (cf + ed)i)                   \\
                                 & = (ace - bde - adf - bcf) + (ade + bce + acf - bdf)i \\
        \Rightarrow        \quad & (\alpha\beta)\lambda = \alpha(\beta\lambda)
    \end{aligned}
$$

\subsection{}
Show that for every $\alpha \in \bm{C}$, there exists a unique $\beta \in \bm{C}$ such that $\alpha + \beta = 0$.
\paragraph{}
We know for every $a \in \bm{R}$, there exists a unique $b \in \bm{R}$ such that $a + b = 0$, more specifically, $b = -a$,
so we let $\alpha = a + bi,\beta = c + di, a,b,c,d\in \bm{R}$.
Let $c = -a, d = -b$, there will be $\alpha + \beta = (a + c) + (b + d)i = 0$, so the unique one is $\beta = - a - bi = -\alpha$.

\subsection{}
Show that for every $\alpha \in \bm{C}$ with $\alpha\neq 0$, there exists a unique $\beta \in \bm{C}$ such that $\alpha\beta = 1$.
\paragraph{}
From the Problem 1, we know the unique one of $\alpha = a + bi, a,b\in \bm{R}$ is
$$
    \beta = \frac{a}{a^2+b^2} - \frac{b}{a^2+b^2}i
$$

\subsection{}
Show that $\lambda(\alpha+\beta)=\lambda\alpha+\lambda\beta$for all $\lambda,\alpha,\beta\in\bm{C}$.
\paragraph{}
Let $\alpha = a + bi, \beta = c + di, \lambda = e + fi$ for all $a,b,c,d,e,f \in \bm{R}$, we have
$$
    \begin{aligned}
        \lambda(\alpha+\beta)
                          & = (e+fi)(a+bi+c+di)                              \\
                          & =(e+fi)((a+c)+(b+d)i)                            \\
                          & =(ae+ce-bf-df)+(af+cf+be+de)i                    \\
        \lambda\alpha+\lambda\beta
                          & =(e+fi)(a+bi)+(e+fi)(c+di)                       \\
                          & =(ae-bf)+(af+be)i+(ce-df)+(cf+de)i               \\
                          & =(ae+ce-bf-df)+(af+cf+be+de)i                    \\
        \Rightarrow \quad & \lambda(\alpha+\beta)=\lambda\alpha+\lambda\beta
    \end{aligned}
$$

\subsection{}
Find $x\in \bm{R}^4$such that
$$
    (4,-3,1,7)+2x=(5,9,-6,8).
$$
\paragraph{}
Let $x = (a,b,c,d), a,b,c,d\in \bm{R}$, and according to the equation above, we have
$$
    \begin{cases}
        4+2a=5  \\
        -3+2b=9 \\
        1+2c=-6 \\
        7+2d=8
    \end{cases}\Rightarrow
    \begin{cases}
        a = 5 \\
        b=6   \\
        c=-5  \\
        d = 5
    \end{cases}\Rightarrow
    x = (5,6,-5,5)
$$

\subsection{}
Explain why there does not exist $\lambda \in \mathbf{C}$ such that
\[
    \lambda(2-3 i, 5+4 i,-6+7 i)=(12-5 i, 7+22 i,-32-9 i).
\]
\paragraph{}
From the equation above, we have
\[
    \begin{cases}
        \lambda(2-3i)=12-5i \\
        \lambda(5+4i)=7+22i \\
        \lambda(-6+7i)=-32-9i
    \end{cases}
\]
For the first equation, we have $\lambda = 3+2i$.
Also for the second equation, we need $\lambda = 3+2i$.
But for the third equation, when $\lambda = 3+2i$,
the left side will be $(3+2i)(-6+7i)=-32+9i\neq-32-9i$.
So here exists conflicts.

\subsection{}
Show that $(x+y)+z=x+(y+z)$ for all $x, y, z \in \mathbf{F}^{n}$.
\paragraph{}
\[
    \begin{aligned}
        (x+y)+z & = ((x_1,x_2,\ldots,x_n)+(y_1,y_2,\ldots,y_n))+(z_1,z_2,\ldots,z_n) \\
                & = (x_1+y_1,x_2+y_2,\ldots,x_n+y_n)+(z_1,z_2,\ldots,z_n)            \\
                & = (x_1+y_1+z_1,x_2+y_2+z_2,\ldots,x_n+y_n+z_n)                     \\
                & = (x_1+(y_1+z_1),x_2+(y_2+z_2),\ldots,x_n+(y_n+z_n))               \\
                & = (x_1,x_2,\ldots,x_n)+(y_1+z_1,y_2+z_2,\ldots,y_n+z_n)            \\
                & = (x_1,x_2,\ldots,x_n)+((y_1,y_2,\ldots,y_n)+(z_1,z_2,\ldots,z_n)) \\
                & = x+(y+z)                                                          \\
    \end{aligned}
\]

\subsection{}
Show that $(a b) x=a(b x)$ for all $x \in \mathbf{F}^{n}$ and all $a, b \in \mathbf{F}$.
\paragraph{}
\[
    \begin{aligned}
        (a b) x & = (ab)(x_1,x_2,\ldots,x_n)         \\
                & = (abx_1,abx_2,\ldots,abx_n)       \\
                & = (a(bx_1),a(b_x2),\ldots,a(bx_n)) \\
                & = a(bx_1,bx_2,\ldots,bx_n)         \\
                & = a(bx)
    \end{aligned}
\]

\subsection{}
Show that $1 x=x$ for all $x \in \mathbf{F}^{n}$.
\paragraph{}
\[
    1x = 1(x_1,x_2,\ldots,x_n) = (1x_1,1x_2,\ldots,1x_n) = (x_1,x_2,\ldots,x_n) =x
\]

\subsection{}
Show that $\lambda(x+y)=\lambda x+\lambda y$   for all  $\lambda \in \mathbf{F}$  and all  $x, y \in \mathbf{F}^{n}$.
\paragraph{}
\[
    \begin{aligned}
        \lambda(x+y) & =\lambda((x_1,x_2,\ldots,x_n)+(y_1,y_2,\ldots,y_n))                                        \\
                     & =\lambda(x_1+y_1,x_2+y_2,\ldots,x_n+y_n)                                                   \\
                     & =(\lambda(x_1+y_1),\lambda(x_2+y_2),\ldots,\lambda(x_n+y_n))                               \\
                     & =(\lambda x_1+\lambda y_1,\lambda x_2+\lambda y_2,\ldots,\lambda x_n+\lambda y_n)          \\
                     & =(\lambda x_1,\lambda x_2,\ldots,\lambda x_n)+(\lambda y_1,\lambda y_2,\ldots,\lambda y_n) \\
                     & =\lambda x+\lambda y
    \end{aligned}
\]

\subsection{}
Show that $(a+b) x=a x+b x$ for all $a, b \in \mathbf{F}$ and all $x \in \mathbf{F}^{n}$.
\paragraph{}
\[
    \begin{aligned}
        (a+b) x & =(a+b)(x_1,x_2,\ldots,x_n)                       \\
                & =((a+b)x_1,(a+b)x_2,\ldots,(a+b)x_n)             \\
                & =(ax_1+bx_1,ax_2+bx_2,\ldots,ax_n+bx_n)          \\
                & =(ax_1,ax_2,\ldots,ax_n)+(bx_1,bx_2,\ldots,bx_n) \\
                & =ax+bx
    \end{aligned}
\]

\section{}

\subsection{}
Prove that $-(-v)=v$ for every $v \in V$.
\paragraph{}
By the definition of additive inverse, we have $w + (-w) = 0$.
The unique additive inverse of $v$ is $-v$ and for $-v$ the unique one is
$-(-v)$. So we get $-(-v) = v$ for every $v \in V$.

\subsection{}
Suppose $a \in \mathbf{F}, v \in V,$ and $a v=0$.Prove that $a=0$ or $v=0$.
\paragraph{}
If $a = 0$, then we are done. If $a \neq 0$, we have $ \frac{1}{a} (av) = \frac{1}{a}(0) \Rightarrow v = 0$.
What should be pointed is that the 0 on the right side of the equation is a vector, so we do depend on the
rule that a number times the vector 0 will be vector 0.

\subsection{}
Suppose $v, w \in V .$ Explain why there exists a unique $x \in V$ such that $v+3 x=w$.
\paragraph{}
Adding $-w$ and then multiplying $\frac{1}{3}$ to the both sides of the equation, we have
\[
    x + \frac{1}{3}(v - w) = 0
\]
Suppose $y = \frac{1}{3}(v - w)$ and we have $x + y = 0$. We know $-y$ is the unique additive
inverse of $y$, so we get $x = -y$. And that means for every $y$, $x$ is unique. So we are Done.


\subsection{}
The empty set is not a vector space. The empty set fails to satisfy only one of the requirements listed in $1.19$. Which one?
\paragraph{}
Every one, I think. Because the empty set has no element.

\subsection{}
Show that in the definition of a vector space $(1.19),$ the additive inverse condition can be replaced with the condition that
\[
    0 v=0 \text { for all } v \in V
\]
Here the 0 on the left side is the number 0 , and the 0 on the right side is the additive identity of $V$. (The phrase "a condition can be replaced" in a definition means that the collection of objects satisfying the definition is unchanged if the original condition is replaced with the new condition.)
\paragraph{}
We can easily get the equation below.
\[
    0v = (1 + (-1))v = 1v + (-1v) = v + (-v)
\]
So from $0v = 0$, we have $v + (-v) = 0$. This means the additive inverse condition
can be replace with this condition.

\subsection{}
Let $\infty$ and $-\infty$ denote two distinct objects, neither of which is in $\mathbf{R}$. Define an addition and scalar multiplication on $\mathbf{R} \cup\{\infty\} \cup\{-\infty\}$ as you could guess from the notation. Specifically, the sum and product of two real numbers is as usual, and for $t \in \mathbf{R}$ define
\[
    \begin{aligned}
        t \infty=\left\{\begin{array}{ll}
            -\infty & \text { if } t<0, \\
            0       & \text { if } t=0, \\
            \infty  & \text { if } t>0,
        \end{array}\right. & t(-\infty)=\left\{\begin{array}{ll}
            \infty  & \text { if } t<0 \\
            0       & \text { if } t=0 \\
            -\infty & \text { if } t>0
        \end{array}\right. \\
        t+\infty=\infty+t                                 & =\infty,                                            \\
        t+(-\infty)=(-\infty)+t                           & =-\infty                                            \\
        \infty+\infty                                     & =\infty,                                            \\
        (-\infty)+(-\infty)                               & =-\infty,                                           \\
        \infty+(-\infty)                                  & =0
    \end{aligned}
\]
Is $\mathbf{R} \cup\{\infty\} \cup\{-\infty\}$ a vector space over $\mathbf{R} ?$ Explain.
\paragraph{}

No. If it is a vector space over $\mathbf{R}$, for example, we have
\[
    \infty = (2 - 1)\infty = 2\infty + (-1)\infty = \infty + (-\infty) = 0
\]
But this is impossible!

\section{}
\subsection{}
For each of the following subsets of $\mathbf{F}^{3},$ determine whether it is a subspace of $\mathbf{F}^{3}$\\
(a) $\quad\left\{\left(x_{1}, x_{2}, x_{3}\right) \in \mathbf{F}^{3}: x_{1}+2 x_{2}+3 x_{3}=0\right\}$ \\
(b) $\quad\left\{\left(x_{1}, x_{2}, x_{3}\right) \in \mathbf{F}^{3}: x_{1}+2 x_{2}+3 x_{3}=4\right\}$ \\
(c) $\quad\left\{\left(x_{1}, x_{2}, x_{3}\right) \in \mathbf{F}^{3}: x_{1} x_{2} x_{3}=0\right\}$ \\
(d) $\quad\left\{\left(x_{1}, x_{2}, x_{3}\right) \in \mathbf{F}^{3}: x_{1}=5 x_{3}\right\}$
\paragraph{}
(a) Let $U = \left\{\left(x_{1}, x_{2}, x_{3}\right) \in \mathbf{F}^{3}: x_{1}+2 x_{2}+3 x_{3}=0\right\}$.
\subparagraph{}
1) Obviously, $(0,0,0) \in U$.
\subparagraph{}
2) Suppose $u = (x_1,x_2,x_3), v = (y_1,y_2,y_3) \in U$ and we have
\[
    \begin{aligned}
                    &
        \begin{cases}
            x_1 + 2x_2 + 3x_3 = 0 \\
            y_1 + 2y_2 + 3y_3 = 0
        \end{cases}        \\
        \Rightarrow & \quad
        (x_1+y_1)+2(x_2+y_2)+3(x_3+y_3)=0 \\
        \Rightarrow & \quad
        u+v=(x_1+y_1,x_2+y_2,x_3+y_3) \in U
    \end{aligned}
\]
\subparagraph{}
3) Suppose $a\in F, u = (x_1,x_2,x_3)\in U$ and we have
\[
    \begin{aligned}
                    & \quad x_1+2x_2+3x_3=0 \\
        \Rightarrow & \quad
        a(x_1+2x_2+3x_3)=ax_1+2ax_2+3ax_3=0 \\
        \Rightarrow & \quad
        au = (ax_1,ax_2,ax_3) \in U
    \end{aligned}
\]
Because $U$ is a nonempty subset of $F^3$ that is closed
under addition and scalar multiplication, $U$ is a subspace of $F^3$.
\newline
\newline
(b) Let $U = \left\{\left(x_{1}, x_{2}, x_{3}\right) \in \mathbf{F}^{3}: x_{1}+2 x_{2}+3 x_{3}=4\right\}$.
Simply, we know $(0,0,0) \notin U$, so $U$ is not a subspace of $F^3$.
\newline
\newline
(c) Let $U = \left\{\left(x_{1}, x_{2}, x_{3}\right) \in \mathbf{F}^{3}: x_{1} x_{2} x_{3}=0\right\}$.
Obviously, $(0,0,0) \in U$. But $U$ is not closed under addition and scalar multiplication, so $U$ is not a subspace of $F^3$.
\newline
\newline
(d) Let $U = \left\{\left(x_{1}, x_{2}, x_{3}\right) \in \mathbf{F}^{3}: x_{1}=5 x_{3}\right\}$.
This format is the same with (a). So it is a subspace of $F^3$, too.

\subsection{}
Verify all the assertions in Example 1.35.
\paragraph{}

(a) If this set is a subspace of $\mathbf{F}^4$, then $(0,0,0,0)\in \mathbf{F}^4$, then $0=5\cdot 0+b$.
Hence $b=0$. (b) (c) and (d) is similar to Problem 3 and 4.
Now let us consider (e).
\newline
\newline
Denote the set of all sequences of complex numbers with limit $0$ by $A$.
\newline

1) Additive identity: it is clear that $(0,0,\cdots)\in A$.
\newline

2) Closed under addition: if $(a_1,a_2,\cdots),(b_1,b_2,\cdots)\in A$, then
\[
    \lim_{n\to\infty}a_n=0\quad\text{and}\quad\lim_{n\to\infty}b_n=0.
\]
It is easy to see
\[
    \lim_{n\to\infty}(a_n+b_n)=\lim_{n\to\infty}a_n+\lim_{n\to\infty}b_n=0+0=0.
\]
This means $(a_1+b_1,a_2+b_2,\cdots)=(a_1,a_2,\cdots)+(b_1,b_2,\cdots)\in A$.
\newline

3) Closed under scalar multiplication: if $(a_1,a_2,\cdots)\in A$, then
\[
    \lim_{n\to\infty}a_n=0.
\]
For any $\lambda\in \mathbf{C}$, it is easy to see
\[
    \lim_{n\to\infty}(\lambda a_n)=\lambda\lim_{n\to\infty} a_n=\lambda 0=0.
\]
This means $\lambda(a_1,a_2,\cdots)=(\lambda a_1,\lambda a_2,\cdots)\in A$.

\subsection{}
Show that the set of differentiable real-valued functions $f$ on the interval (-4,4) such that $f^{\prime}(-1)=3 f(2)$ is a subspace of $\mathbf{R}^{(-4,4)}$
\paragraph{}
Denote the set of differentiable real-valued functions $f$ on the interval $(-4,4)$ such that $f'(-1)=3f(2)$ by $V$.

Additive identity: it is clear that the constant function $f\equiv 0$ is contained in $V$.

Closed under addition: if $f,g\in V$, then $f$ and $g$ are differentiable real-valued functions.
So is $f+g$. Moreover,
\[
    (f+g)'(-1)=f'(-1)+g'(-1)=3f(2)+3g(2)=3(f(2)+g(2))=3(f+g)(2).
\]
This concludes $V$ is closed under addition.

Closed under scalar multiplication: if $f\in V$, for any $\lambda\in \mathbf{R}$, then $f$ is differentiable real-valued functions.
So is $\lambda f$. Moreover,
\[
    (\lambda f)'(-1)=\lambda f'(-1)=\lambda (3f)(2)=3(\lambda f)(2).
\]
This deduces $V$ is closed under scalar multiplication.

\subsection{}
Suppose $b \in \mathbf{R} .$ Show that the set of continuous real-valued functions $f$ on the interval [0,1] such that $\int_{0}^{1} f=b$ is a subspace of $\mathbf{R}^{[0,1]}$ if and only if $b=0$
\paragraph{}
Denote the set of continuous real-valued functions $f$ on the interval $[0,1]$ such that $\int_0^1f=b$ by $V_b$.

If $V_b$ is a subspace of $\mathbf{R}^{[0,1]}$, then for any $f \in V_b$, we have $\int_0^1f=b$.
And we also have $kf\in V_b$ for any $k\in \mathbf{R}$. Hence
\[
    b = \int_0^1(kf) = k\int_0^1f = kb
\]
for all $k\in \mathbf{R}$, and this happens if and only if $b = 0$.

Now if $b = 0$, then for any $f,g\in V_0$ and $\lambda \in \mathbf{R}$. We have
\[
    \int_0^1(f+g) = \int_0^1f+\int_0^1g = 0 + 0 = 0
\]
and $f+g$ is continuous real-valued functions since $f$ and $g$ are.
So $V_0$ is closed under addition.

Similarly,
\[
    \int_0^1(\lambda f) = \lambda\int_0^1 f = k0 = 0
\]
and $\lambda f$ is continuous real-valued functions since $f$ is.
So $V_0$ is closed under scalar multiplication.

At last, the constant function $f \equiv 0 \in V_0$, which is also the additive identity in $\mathbf{R}^{[0,1]}$.

Hence $V_0$ is a subspace of $\mathbf {R}^n$.

\subsection{}
Is $\mathbf{R}^{2}$ a subspace of the complex vector space $\mathbf{C}^{2} ?$
\paragraph{}
No. $i \in C$ and $(1, 1)\in \mathbf{R}^{2}$, but $i(1,1) = (i,i)\notin \mathbf{R}^{2}$.

\subsection{}
(a) $\quad$ Is $\left\{(a, b, c) \in \mathbf{R}^{3}: a^{3}=b^{3}\right\}$ a subspace of $\mathbf{R}^{3} ?$\\
(b) $\quad$ Is $\left\{(a, b, c) \in \mathbf{C}^{3}: a^{3}=b^{3}\right\}$ a subspace of $\mathbf{C}^{3} ?$
\paragraph{}
(a) $\left\{(a, b, c) \in \mathbf{R}^{3}: a^{3}=b^{3}\right\} = \left\{(a, b, c) \in \mathbf{R}^{3}: a=b\right\}$
It's obviously a subspace of $\mathbf{R}^3$.

(b) $\left\{(a, b, c) \in \mathbf{C}^{3}: a^{3}=b^{3}\right\} = \left\{(a, b, c) \in \mathbf{C}^{3}: a=b\ \text{or}\ a = \frac{1\pm \sqrt{3}i}{2}b\right\} = A$
So we have $ x = (\frac{1+\sqrt{3}i}{2},1,0) \in A $ and $ y = (\frac{1-\sqrt{3}i}{2},1,0) \in A $ but $x + y = (1, 2, 0) \notin A$.

\subsection{}
Give an example of a nonempty \textbf{subset} $U$ of $\mathbf{R}^{2}$ such that $U$ is closed under addition and under taking additive inverses (meaning $-u \in U$ whenever $u \in U$ ), but $U$ is not a \textbf{subspace} of $\mathbf{R}^{2}$
\paragraph{}
Suppose $U = \{(x,y)\in \mathbf{R}^2 : x , y\in \mathbf{Z}\}$. It's easy to proof that $U$ is closed under addition and under taking additive inverses.
The most important is that $\frac{1}{2}\in R$ and $(1,1)\in U$ but $\frac{1}{2}(1,1) = (\frac{1}{2},\frac{1}{2}) \notin U$

\subsection{}
Give an example of a nonempty subset $U$ of $\mathbf{R}^{2}$ such that $U$ is closed under scalar multiplication, but $U$ is not a subspace of $\mathbf{R}^{2}$
\paragraph{}
Suppose $U = \{(x,y)\in \mathbf{R}^2 : x = 0 \text{ or } y = 0\}$, then $U$ is nonempty. $(0,y),(x,0)\in U$, then for $\lambda \in \mathbf{R}$, we have
$\lambda(0,y)=(0,\lambda y)\in U$. Similarly, we have $\lambda(x,0)=(\lambda x,0)\in U$, hence $U$ is under scalar multiplication. However, $(1,0), (0,1)\in U$ while
$(0,1)+(1,0)=(1,1)\notin U$. That means $U$ is not a subspace of $\mathbf{R}^2$.

\subsection{}
A function $f: \mathbf{R} \rightarrow \mathbf{R}$ is called periodic if there exists a positive number $p$ such that $f(x)=f(x+p)$ for all $x \in \mathbf{R} .$ Is the set of periodic functions from $\mathbf{R}$ to $\mathbf{R}$ a subspace of $\mathbf{R}^{\mathbf{R}} ?$ Explain.
\paragraph{}
Denote the set of periodic functions from $\mathbf{R}$ to $\mathbf{R}$ by $S$.
We have $f(x) = \cos(x) \in S$ and $g(x) = \sin\sqrt{2}(x) \in S$, then $h(x) = f(x) + g(x) = \cos x + \sin\sqrt{2}x$.
Assume that there exists a positive number $p$ such that $h(x) = h(x+p)$ for all $x\in R$, then we have
\[
    \begin{aligned}
        1 = h(0) = h(p) = h(-p)
         & \Rightarrow
        \begin{cases}
            \cos p + \sin\sqrt{2}p = \cos (-p) + \sin\sqrt{2}(-p) \\
            \cos p + \sin\sqrt{2}p = 1
        \end{cases} \\
         & \Rightarrow
        \begin{cases}
            \sin\sqrt{2}p = 0 \\
            \cos p = 1
        \end{cases}
        \Rightarrow
        \begin{cases}
            p = 2k\pi \\
            \sqrt{2}p=l\pi
        \end{cases}(k,l\in \mathbf{Z})
    \end{aligned}
\]
Hence,
\[
    \sqrt{2} = \frac{l\pi}{2k\pi} = \frac{l}{2k} \in \mathbf{Q}
\]
which is impossible. Therefore we get the conclusion.

\subsection{}
Suppose $U_{1}$ and $U_{2}$ are subspaces of $V .$ Prove that the intersection $U_{1} \cap U_{2}$ is a subspace of $V$
\paragraph{}
1. Additive identity: By definition, we know $0\in U_1$ and $0\in U_2$, hence $0\in U_1 \cap U_2$.

2. Closed under addition: If $x, y \in U_1 \cap U_2$ then $x, y\in U_1$ and $x, y\in U_2$, hence
$x + y \in U_1$ and also $x + y \in U_2$, then $x+y\in U_1 \cap U_2$.

3. Closed under scalar multiplication: If $x\in U_1\cap U_2$, then $x \in U_1$ and $x \in U_2$. Then for any $\lambda in \mathbf{F}$,
we have $\lambda x\in U_1$ and $\lambda x \in U_2$ since $U_1, U_2$ is closed under scalar multiplication. Therefore $\lambda x \in U_1 \cap U_2$.

\subsection{}
Prove that the intersection of every collection of subspaces of $V$ is a subspace of $V$
\paragraph{}
Assume $U_i$ are subspaces of $V$, where $i\in I$. Now we will show $\cap_{i\in I}U_i$ is a subspace of $V$.

Additive identity: by definition $0\in U_i$ for every $i\in I$, hence $0\in \cap_{i\in I}U_i$.

Closed under addition: if $x\in \cap_{i\in I}U_i$ and $y\in \cap_{i\in I}U_i$, then for any given $i\in I$,
we have $x\in U_i$ and $y\in U_i$, hence $x+y\in U_i$ for $U_i$ is closed under addition. Therefore $x+y\in \cap_{i\in I}U_i$.

Closed under scalar multiplication: if $x\in \cap_{i\in I}U_i$, then $x\in U_i$ for any given $i\in I$. Then for any $\lambda\in\mathbf{F}$,
we have $\lambda x\in U_i$ since $U_i$ is closed under scalar multiplication. Therefore $\lambda x\in \cap_{i\in I}U_i$.

\subsection{}
Prove that the union of two subspaces of $V$ is a subspace of $V$ if and only if one of the subspaces is contained in the other.
\paragraph{}
Suppose $U,W$ are both subspaces of $V$.

If $U \cup W $ is a subspace of $V$, moreover $U \not\subset W$ and $W\not\subset U$. Consider $u \in U - W$ and $w \in W - U$, then $u + w\in U \cup W$.
Hence $u+w\in U \text{ or } \in W$. If $u+w \in U$, then $w = (u+w) - u \in U$. We get a contradiction. If $u+w \in W$, then $u = (u+w)-w\in W$.
We get another contradiction. Hence if $U\cup W$is a subspace of $V$, we must have $U\subset W$ or $W\subset U$.

If $U\subset W$ or $W\subset U$, without loss of generality, we can assume $U\subset W$, then $U\sup W = W$ is obviously a subspace of $V$.

\subsection{}
Prove that the union of three subspaces of $V$ is a subspace of $V$ if and only if one of the subspaces contains the other two. [This exercise is surprisingly harder than the previous exercise, possibly because this exercise is not true if we replace $\mathbf{F}$ with a field containing only two elements.]
\paragraph{}
TODO

\subsection{}
Verify the assertion in Example 1.38
\paragraph{}
It is clear that $U$ and $W$ are subspaces of $\mathbf{F}^4$.

Now assume that $(x_1,x_1,y_1,y_1)\in U$ and $(x_2,x_2,x_2,y_2)\in W$,
then
\[
    \begin{aligned}
          & (x_1,x_1,y_1,y_1)+(x_2,x_2,x_2,y_2)                                     \\
        = & (x_1+x_2,x_1+x_2,y_1+x_2,y_1+y_2) \in \{(x,x,y,z):x,y,z\in\mathbf{F}\}.
    \end{aligned}
\]

Hence $U+W\subset \{(x,x,y,z):x,y,z\in\mathbf{F}^4\}$. For any $x,y,z\in\mathbf{F}$, we have $(0,0,y-x,y-x)\in U$ and $(x,x,x,z+x-y)\in W$.

However,
\[
    (x,x,y,z)=(0,0,y-x,y-x)+(x,x,x,z+x-y)\in U+W
\]

Hence $\{(x,x,y,z):x,y,z\in\mathbf{F}^4\}\subset U+W$.

Combining this with previous argument, it follows that $U+W=\{(x,x,y,z):x,y,z\in\mathbf{F}\}$.

\subsection{}
Suppose $U$ is a subspace of $V .$ What is $U+U$ ?
\paragraph{}
It will still be $U$. Because $U$ is a subspace of $V$, hence closed under addition. Therefore for any $x,y\in U$, we have $x+y\in U$, i.e. $U+U\subset U$.
Note that if $x\in U$, then $x=x+0\in U+U$, hence $U\subset U+U$. So $U+U=U$

\subsection{}
Is the operation of addition on the subspaces of $V$ commutative? In other words, if $U$ and $W$ are subspaces of $V$, is $U+W=W+U ?$
\paragraph{}
For $x\in U$ and $y\in W$, because $U$ is a subspace of $V$, hence closed under addition in $V$ is commutative, we have $x+y=y+x\in W+U$.
This implies $U+W\subset W+U$. Similarly, we have $W+U \subset U+W$. Hence $U+W=W+U$.

\subsection{}
Is the operation of addition on the subspaces of $V$ associative? In other words, if $U_{1}, U_{2}, U_{3}$ are subspaces of $V,$ is
\[
    \left(U_{1}+U_{2}\right)+U_{3}=U_{1}+\left(U_{2}+U_{3}\right) ?
\]
\paragraph{}
Note that in $V$, we have $(x+y)+z=x+(y+z)$. Hence this similar to Problem 16. Let $x_i\in U_i, i=1,2,3$, then
\[
    (x_1+x_2)+x_3=x_1+(x_2+x_3)\in U_1 +(U_2+U_3)
\]
Since every element in $(U_1+U_2)+U_3$ can be expressed as the form $(x_1+x_2)+x_3$. It follows that $(U_1+U_2)+U_3\subset U_1+(U_2+U_3)$.
Similarly, we also have $U_1+(U_2+U_3)\subset (U_1+U_2)+U_3$. Hence $(U_1+U_2) + U_3=U_1+(U_2+U_3)$.

\subsection{}
Does the operation of addition on the subspaces of $V$ have an additive identity? Which subspaces have additive inverses?
\paragraph{}
Denote the additive identity by $U$. Suppose $W$ is a subspace of $V$. By definition, we have $U + W = W$. This means $U\subset W$ by the similar arguments in
Problem 15 and 16. Hence the only possibility is $U = 0$.

Suppose subspace $W$ of $V$ has additive inverses, then there exists a subspace $S$ of $V$ such that $S+W=0$. This can only happen when $W=0$
since $W\subset W+S$.

\subsection{}
Prove or give a counterexample: if $U_{1}, U_{2}, W$ are subspaces of $V$ such that
\[
    U_{1}+W=U_{2}+W
\]
then $U_{1}=U_{2}$
\paragraph{}
It's easy to give. Suppose $U_1 = V, U_2 \subseteq V, W = V - U_2 \subset V$. Then we have
\[
    U_2 + W = V - U_2 + U_2 = V = U_1 = U_1 + W
\]
But $U1 \neq U_2$.

\subsection{}
Suppose
\[
    U=\left\{(x, x, y, y) \in \mathbf{F}^{4}: x, y \in \mathbf{F}\right\}
\]
Find a subspace $W$ of $\mathbf{F}^{4}$ such that $\mathbf{F}^{4}=U \oplus W$
\paragraph{}
$W = \{(0,z,0,w) \in \mathbf{F}^4: z, w\in \mathbf{F}\}$. For any $(x,y,z,w)\in \mathbf{F}^4$, we have
\[
    (x,y,z,w) = (x,x,z,z) +(0,y-x,0,w-z)\in U+W
\]
since $(x,x,z,z)\in U$ and $(0,y-x,0,w-z)\in W$. We have $\mathbf{F}^4 = U + W$.

Moreover, if $(x,y,z,w) \in U \cap W$, then we must have $x = y$ and $z = w$ since $(x,y,z,w) \in U$.

Similarly, since $(x,y,z,w)\in W$, we have $x = 0$ and $z = 0$. Therefore, $x = y = 0$ and $z = w = 0$, hence $(x,y,z,w) = (0,0,0,0)$.
It follows that $U\cap W = {0}$. Hence $F = U \oplus W$.

\subsection{}
Suppose
\[
    U=\left\{(x, y, x+y, x-y, 2 x) \in \mathbf{F}^{5}: x, y \in \mathbf{F}\right\}
\]
Find a subspace $W$ of $\mathbf{F}^{5}$ such that $\mathbf{F}^{5}=U \oplus W$
\paragraph{}
$W = \{(0,0,a,b,c)\in \mathbf{F}^5: a,b,c \in \mathbf{F}\}$. For any $(x,y,a,b,c)\in \mathbf{F}^5$, we have
\[
    (x,y,a,b,c) = (x,y,x+y,x-y,2x) + (0,0,a-x-y,b-x+y,c-2x)\in U+W
\]
since $(x,y,x+y,x-y,2x)\in U$ and $(0,0,a-x-y,b-x+y,c-2x) \in W$. Therefore $\mathbf{F}^5 = U + W$.

And we will have the same procedure to proof $U \cap W = {0}$ like Problem 20. So $\mathbf{F}^{5}=U \oplus W$.

\subsection{}
Suppose
\[
    U=\left\{(x, y, x+y, x-y, 2 x) \in \mathbf{F}^{5}: x, y \in \mathbf{F}\right\}
\]
Find three subspaces $W_{1}, W_{2}, W_{3}$ of $\mathbf{F}^{5},$ none of which equals $\{0\},$ such that $\mathbf{F}^{5}=U \oplus W_{1} \oplus W_{2} \oplus W_{3}$
\paragraph{}
It's easy to find.
\[
    \begin{aligned}
         & W_1 = \{(0,0,a,0,0) \in \mathbf{F}^5: a \in \mathbf{F}\} \\
         & W_2 = \{(0,0,0,b,0) \in \mathbf{F}^5: b \in \mathbf{F}\} \\
         & W_3 = \{(0,0,0,0,c) \in \mathbf{F}^5: c \in \mathbf{F}\} \\
    \end{aligned}
\]
By the same argument as in Problem 20\&21, we have $\mathbf{F}^5 = U \oplus W_1 \oplus W_2 \oplus W_3$.

\subsection{}
Prove or give a counterexample: if $U_{1}, U_{2}, W$ are subspaces of $V$ such that
\[
    V=U_{1} \oplus W \quad \text { and } \quad V=U_{2} \oplus W
\]
then $U_{1}=U_{2}$
\paragraph{}
Let $V = \mathbf{R}^2, U_1 = \{(x,0)\in \mathbf{R}^2 : x\in \mathbf{R}\}, U_2 = \{(0,y)\in \mathbf{R}^2 : y\in \mathbf{R}\}, W = \{(a,a)\in \mathbf{R}^2: a\in \mathbf{R}\}$.
We have $V = U_1 \oplus W$ and $V = U_2 \oplus W$, but $U_1\neq U_2$.

\subsection{}
A function $f: \mathbf{R} \rightarrow \mathbf{R}$ is called even if
\[
    f(-x)=f(x)
\]
for all $x \in \mathbf{R} .$ \\
A function $f: \mathbf{R} \rightarrow \mathbf{R}$ is called odd if
\[
    f(-x)=-f(x)
\]
for all $x \in \mathbf{R} .$ \\
Let $U_{\mathrm{e}}$ denote the set of real-valued even functions on $\mathbf{R}$ and let $U_{\mathrm{o}}$ denote the set of real-valued odd functions on $\mathbf{R}$. Show that
\[
    \mathbf{R}^{\mathbf{R}}=U_{\mathrm{e}} \oplus U_{\mathrm{o}}.
\]
\paragraph{}
Given $f\in \mathbf{R^R}$ and two other function made by $f$:
\[
    f_e = \frac{f(x) + f(-x)}{2} \quad f_o = \frac{f(x) - f(-x)}{2}
\]
Of course, $f_e,f_o\in \mathbf{R^R}$.

Apparently, $f = f_e + f_o \in U_e + U_o$, so we get $\mathbf{R^R} = U_e + U_o$.

Let $f \in U_e\cap U_o$, then $f(x) = f(-x)$ since $f\in U_e$ and $f(x) = -f(x)$ since $f\in U_o$
for all $x\in \mathbf {R}$. Sum up the two equations, we get $f(x) = 0$ for all $x\in \mathbf{R}$.

Hence $f = 0$, which implies $U_e\cap U_o = \{0\}$ and $\mathbf{R^R} = U_e \oplus U_o$.


\end{document}